\begin{agradecimentos}
\begin{flushright}
	"Quando seu coração está pleno de gratidão,\\ qualquer porta aparentemente fechada,\\ 
	pode ser uma abertura para uma bênção maior." \\
	\ \\
	(Osho)
\end{flushright}
Gostariamos de agradecer primeiramente a nossas famílias, que sempre nos apoiaram em nossas decisões e nos guiaram até nos tornarmos as pessoas que somos hoje.

Gostariamos de agradecer a todos os profissionais do BIR que de certa forma nos ajudaram com ideias, feedbacks, discussões e até mesmo assuntos técnicos abordados. Agradecimentos especiais a Alano, Téo e Gabriel Santos por ideias que foram de suma importância.

Agradacemos especialmente ao professor Oberdan e à Romulo pela compreensão e toda ajuda demonstrada com relação a nossa realização desta monografia. 

Agradecemos ainda ao Centro Universitário SENAI-CIMATEC pelos anos de faculdade que passaram, por toda a infraestrutura disponibilizada, e acima de tudo aos nossos professores e educadores, sem os quais não seria possível estar aqui. Agradecimentos especiais ao professor Guilherme Souza, coordenador do curso de Engenharia Mecânica por todo o apoio e tutoria em momentos decisivos do nosso aprendizado.

Por fim, gostariamos de agradecer ao nosso Orientador, o professor Marco Reis, por sempre ser uma figura que nos estimulou a dar o nosso melhor e por todo o reconhecimento de nossa capacidade. Agradecemos ainda a oportunidade que nos foi dada de abordar o tema proposto a seguir.
\end{agradecimentos}