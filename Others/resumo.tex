\begin{thesisresumo}
O assunto robótica sempre gerou muita polêmica e presença de tabus enormes sobre seus conceitos e possíveis usos. Esta monografia tratará da proposição de uma nova abordagem de ensino, voltada para a robótica, que irá aliar aprendizados prático e teórico ao mesmo tempo que aborda conceitos básicos da robótica aplicada. A abordagem apresentada será baseada em uma junção de metodologias de ensino focadas no aprendizado prático como exemplo o movimento maker, o PBL e o TBL, e a concepção de Vygotsky sobre o aprendizado. O conteúdo teórico foi escrito em linguagem acessível, e disposto em formato de tutoriais e apostilas disponíveis em domínio virtual. O kit físico será dividido em módulos complementares de montagem, tendo como principal diferencial a união da prática com a teoria de forma gradual. O kit de aprendizado promete ser um bom precursor de avanços no ensino da robótica.	

% use de três a cinco palavras-chave

\textbf{Palavras-chave}: Robótica; Ensino; Teoria; Prática.

\end{thesisresumo}
