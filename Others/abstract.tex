\begin{thesisabastract}
Robotics as a matter of discussion, has always raised several debates around whether it is safe, of around it's difficulty. This thesis aims to propose a new approach on teaching robotics, which combines theory and hands-on learning, as well as Vygotsky's view on learning, while addressing basic robotics concepts. This approach will be based on a junction of methodologies which focus on theoretical teaching in addition with methodologies which focus on practical learning. The theoretical content was written in a simplified language and was displayed in a tutorial format and made available in an online environment. The physical kit will be divided in complementary and gradual assembling steps that will result in a differential robot. This new approach focuses on simplifying specific robotics contents while combining theoretical and practical learning, which may serve as a reference for future enhancements into teaching robotics.
\ \\

\textbf{Keywords}: Robotics; Teaching; Theory; Vygotsky.

\end{thesisabastract}
