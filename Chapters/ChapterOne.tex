\chapter{Introdução}
\label{chap:intro}
\begin{flushright}
	"Faça ou não faça, tentativa não há." \\
	\ \\
	(Mestre Yoda)
\end{flushright}

fredinho mama coisinhas.



%--------- NEW SECTION ----------------------
\section{Organização do \thetypework}
\label{section:organizacao}
O documento está organizado em cinco capítulos, seguindo a seguinte estrutura:

\textbf{Capitulo 1 - Introdução}: Faz a contextualização do âmbito no qual a pesquisa proposta
está inserida. Apresenta, portanto, a problemática, objetivos e como este projeto Theoprax de conclusão de curso está estruturado


\textbf{Capítulo 2 - Referencial Teórico}: Apresenta a base teórica necessária para o desenvolvimento do projeto.

\textbf{Capítulo 3 - Metodologia}: Define o método adotado para o desenvolvimento do projeto, explicitando seu fluxo de atividades e premissas necessárias para aplicar a metodologia.

\textbf{Capítulo 4 - Desenvolvimento}: Exibe os procedimentos realizados e resultados obtidos através de testes, unitários e integrados, durante o desenvolvimento do projeto.

\textbf{Capítulo 5 - Conclusão}: Apresenta as conclusões, contribuições e algumas sugestões de atividades de pesquisa a serem desenvolvidas futuramente.


