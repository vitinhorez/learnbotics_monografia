\chapter{Fundamentação Teória}
\label{chap:concep}

\section{Didática}\label{sec:didatica}

Por muito tempo, segundo \cite{saviani}, nas sociedades antigas e medievais, a educação era obtida pela maioria das pessoas através do trabalho, sendo a escola apenas um complemento secundário disponível a grupos seletos da elite. A partir do surgimento da sociedade capitalista e seus processos mercantis que os trabalhadores passaram a precisar de conhecimentos que não se voltavam apenas a técnicas de trabalho manual como cultivo da terra, mas também a relações mercantis e acumulação de capital, ao qual inclui instrumentos de trabalho e, posteriormente, moeda, marcando o início da idade moderna e avanço da ciência.

De acordo com \cite{larchert}, o estudo da didática suma nasceu somente em meados do século XVII, a partir dos estudos pioneiros realizados por Jan Amos Komenský (em latim: Comenius; em português: Comênio), que é considerado pai da didática moderna. Em sua principal obra: Didática Magna, Comenius reuniu os conhecimentos sobre o tema e propôs pela primeira vez um modelo revolucionário de ensino em que os conhecimentos científicos são passados não pela imposição característica da época, mas sim pela satisfação e alegria de ensinar e aprender, segundo \cite{gasparin}. A partir de então, a didática, pedagogia e os modelos de escola passaram a ser ponto de estudo enfatizando a maneira de organizar o ensino e moldada de acordo com as necessidades, exigências e transformações da sociedade ao longo dos anos.

A pedagogia pode ser classificada em quatro diferentes escolas:  Tradicional, Renovada, Tecnicista e Crítica. \cite{larchert} justifica essa classificação de acordo com as características de cada elemento na estrutura didática utilizada, como exposto a seguir.

\subsection{Pedagogia Tradicional}\label{sec:ped_trad}

Do século XVII ao século XIX a didática ficou conhecida como didática tradicional, que se caracteriza por acentuar o ensino humanístico, onde os conteúdos, os procedimentos didáticos e a relação professor-aluno não têm nenhuma relação com o cotidiano do aluno e muito menos com as realidades sociais. Há a predominância da palavra do professor, das regras impostas e do cultivo exclusivamente intelectual \cite{libaneo}.
Segundo \cite{larchert}, sua estrutura organizacional é baseada da seguinte forma:
\begin{quote} - "Homem: compreendido a partir dos ideais de igualdade e liberdade pautados na política liberal.

- Conhecimento: a ciência é a única fonte de conhecimento verdadeiro. O conhecimento enciclopédico é transmitido de uma geração para outra;
	
- Escola: a instituição responsável em preparar os indivíduos para o desempenho dos papéis sociais. Deverá educar para as normas vigentes da sociedade através do desenvolvimento das aptidões individuais, garantindo o preparo intelectual e moral dos alunos;
	
- Método de ensino: instrução organizada na exposição verbal do professor, provocando o acúmulo de conteúdos: 
	
	1º) apresentação da matéria nova de forma clara e completa;
	2º) associação entre os conteúdos antigos e os novos;	
	3º) sistematização e generalização dos conteúdos;
	4º) aplicação de exercícios e testes;

- Professor: é o arquiteto da mente, o dono do saber, a autoridade maior responsável pelo ensino, centro do processo educativo;

- Aluno: receptor dos conteúdos transmitidos e do método aplicado;

- Ensino: propedêutico, sustentado nas cátedras, organizado basicamente pela oratória do professor e pela aula expositiva, entendido como a “arte de ensinar” e toda ênfase está na teoria;
	
- Aprendizagem: memorização de conteúdos, acúmulo de saberes transmitidos pelo professor e repetidos nos livros;
	
- Avaliação da aprendizagem: individual, oral ou escrita".	
\end{quote}
Essa pedagogia dominou o Brasil até a década de 1930, mas ainda pode ser visto no cenário atual.

\subsection{Pedagogia Renovada}\label{sec:ped_renov}

A partir da segunda metade do século XX, estudiosos desenvolveram a ideia de que a aprendizagem verdadeira é aquela que nasce dos aspectos sociais, emocionais e cognitivos do aluno, não considerando o ensino como um processo de transmissão, pelo professor aos alunos, de conteúdos prontos, mas uma facilitação da aprendizagem, com o professor na função de estimular o aprendiz a aprender \cite{larchert}. Ela acentua, igualmente, o sentido da cultura como desenvolvimento das aptidões individuais, mas a educação é um processo interno, não externo; ela parte das necessidades e interesses individuais necessários para a adaptação ao meio \cite{libaneo}.

Sua estrutura organizacional é baseada da seguinte forma, segundo \cite{larchert}:
\begin{quote}- "Homem: compreendido a partir da sua existência, indivíduo único no mundo, vive e interage em um mundo dinâmico;

- Conhecimento: produto da existência e da experiência que deverá ser compreendido e socializado;

- Escola: instituição organizadora e articuladora entre as estruturas cognitivas do indivíduo e as estruturas do meio ambiente;

- Método de ensino: aprender experimentando, aprender a aprender;
	
- Professor: facilitador da aprendizagem e do desenvolvimento humano;

- Aluno: o aprendiz é o centro do processo educativo;

- Ensino: orientação individual para o aprender fazendo, através de experimentação, pesquisas, dinâmicas de grupo e oficinas;
	
- Aprendizagem: atividade de descoberta, autoaprendizagem;

- Avaliação da aprendizagem: autoavaliação".
\end{quote}

\subsection{Pedagogia Tecnicista}\label{sec:ped_tecni}

No Brasil, entre 1950 e 1970, ocorre o desenvolvimento industrial e tecnológico, cenário propício para o desenvolvimento do tecnicismo educacional, inspirado nas teorias behavioristas, cujas unidades de análises são respostas e estímulos. O indivíduo, motivado por estímulos planejados com rigor, responderá satisfatoriamente aos comandos propostos. Esta abordagem atende às exigências da sociedade capitalista, industrial e tecnológica, pois, para a sua sobrevivência, a massa de trabalhadores precisa de objetivos rigorosamente planejados, executados e controlados \cite{larchert}.
O autor baseia a estrutura organizacional da seguinte forma:
\begin{quote}- "Homem: produto do meio social;
	
- Conhecimento: resultado da experiência planejada, baseada nos princípios científicos;

- Escola: responsável por qualificar mão de obra para o mercado de trabalho;

- Método de ensino: processo de condicionamento através do estímulo – resposta;

- Professor: especialista de uma determinada área é um técnico capacitado para reproduzir com os alunos as dinâmicas aprendidas;

- Aluno: tem o papel de receber, fixar e repetir as técnicas e seus conteúdos;

- Ensino: diretivo e instrução programada;

- Aprendizagem: fixação do programa aplicado, dos conteúdos decorados e da repetição da técnica;

- Avaliação da aprendizagem: verificação dos resultados dos objetivos propostos".
\end{quote}

\subsection{Pedagogia Crítica}\label{sec:ped_crit}

A partir de meados da década de 70, a sociedade brasileira passou a ter uma visão crítica acentuada quanto ao cenário político autoritário, levando esse pensamento crítico a contestar diversas áreas, incluindo o modelo didático utilizado. De acordo com \cite{martins}, essas discussões deram origem a uma didática que traz como princípio norteador a abordagem crítica da educação, influencia o potencial social dos alunos e professores através de debates, diálogos críticos e considera o ensino nas suas múltiplas dimensões: social, afetivo, cognitivo, motor e político. Estudiosos como Paulo Freire ganharam espaço com proposições de um ensino mais dinâmico e menos alienado, voltado para a libertação do homem. Para \cite{freire}, o diálogo aberto, igualitário e ativo em todos os interlocutores é o principal caminho para a real comunicação, sendo o único capaz de dissipar e gerar conhecimento com eficácia.

De acordo com \cite{larchert}, sua estrutura organizacional é baseada da seguinte forma:
\begin{quote}- "Homem: cidadão e agente de transformação social;
	
- Conhecimento: socialmente referenciado, reflexivo e crítico;

- Escola: construtora de conhecimento crítico que busca a transformação social;

- Método de ensino: dialético, parte da experiência do aluno e do professor, confrontando-a com o saber sistematizado, pautado na dialogicidade, o diálogo como produtor de conhecimento e de emancipação do cidadão;

- Professor: mediador, orientador e agente de mudança social;

- Aluno: aprendiz participativo e crítico;

- Ensino: organização de experiências destacando os conhecimentos da ciência para explicá-las criticamente;

- Aprendizagem: desenvolvimento de estruturas cognitivas e sociais para a emancipação do aluno e do professor;

- Avaliação da aprendizagem: avaliar para mudar; autoavaliação".
\end{quote}
\section{Didática na Sociedade Contemporânea}\label{sec:didatica_sociedade_cont}
A Didática, de acordo com as definições de \cite{libaneo}, é uma das disciplinas da pedagogia, sendo responsável por estudar e investigar, teórica e praticamente, os objetivos, conteúdos, meios e as condições do processo de ensino, a fim de educar o indivíduo no âmbito social. Por ser dependente da sociedade ao qual o indivíduo está inserido, os modelos didáticos estão em constante transformação para adequar a educação ao modelo social vigente em cada época. No entanto, segundo \cite{al-mufti}, os modelos atuais de educação não estão atendendo ao rápido crescimento demográfico e tecnológico do século XXI, sendo necessário uma adequação mundial a uma sociedade globalizada, tecnológica e superpovoada. O perfil de aluno contemporâneo é de um indivíduo que cresce em um ambiente sem fronteiras onde parece não haver limites para a velocidade na troca de informações, que ocorre em tempo real, e onde a internet  permite obter qualquer tipo de informação na palma das mãos e de onde estiver. Qualquer questão pode ser respondida com uma simples pergunta no Google, transferindo a fonte de aprendizado e modificando o papel da escola. Para atender ao perfil deste aluno é preciso ensiná-lo a buscar o próprio conhecimento através de informações corretas, guiá-lo em meio a infinidade de informação equivocada que ele pode encontrar na web. Parte dessa adequação se passa por abdicar dos modelos didáticos defasados que ainda persistem em dominar as escolas ao longo do globo, incluindo o Brasil, onde existem vestígios do sistema tradicional; e preparar o aluno para se adequar às transformações cada vez mais frequentes do século XXI. Por isso é cada vez mais coerente pensamentos como o de \cite{freire}, no qual diz que estamos numa sociedade em transição precisando romper com o mal da alienação e investir numa educação crítica que adapta e integra o homem ao mundo invés de acomodá-lo e transformá-lo. 

\subsection{Ensino de Novas Tecnologias}\label{sec:ens_novas_tecn}
Na sociedade do século XXI, de acordo com \cite{benitti}, as pessoas estão imersas desde a infância em ambientes com infinidade de informações, de tecnologia avançada e dinâmica que ultrapassam os limites dos laboratórios e englobam as casas, carros, celulares, computadores e todo o entorno, sendo  muito utilizada, mas pouco conhecida pela maioria da população. Este cenário tem expectativa de severidade maior com o desenvolvimento da tecnologia, a chegada da indústria 4.0 e internet das coisas, sendo necessário uma revolução no sistema de ensino para adequar o homem às atividades e profissões do futuro. Uma forma de viabilizar isso é através da robótica educativa, na qual o aluno é estimulado a desenvolver a sua criatividade, senso crítico, capacidade de elaborar hipóteses, investigar soluções, tirar conclusões e resolver problemas enquanto cria afinidade com conceitos que estão inseridos no seu cotidiano, mas passam despercebidos por boa parte das pessoas. Para tal utiliza-se de softwares didáticos, kits intuitivos e metodologias de ensino baseadas na pedagogia-crítica, onde o aluno é participativo e livre, enquanto o professor é um mediador do conhecimento. Essas metodologias valorizam o trabalho em equipe e desenvolvem a busca própria de aprendizado por mera curiosidade e prazer através de desafios e metodologias DIY (do inglês: Do It Yourself; em português: faça você mesmo) e STEAM (do inglês: science, technology, engineering and Mathematics; em português: ciência, tecnologia, engenharia e matemática), como pode ser visto nos principais kits de ensino de robótica do mercado, por exemplo: Modelix Robotics, Robomind, Mini Maker, Mini Bots e Lego.

\subsection{Movimento STEM}\label{sec:mov_stem}
De acordo com \cite{pugliese}, STEM education (ou educação STEM, em português) não é exatamente uma metodologia de ensino, mas sim um movimento, resultado de uma transformação maior que muitos sistemas educacionais vêm passando globalmente, decorrente da revolução tecnológica e consequente necessidade de inovação nos modelos de ensino. Este movimento nasceu na década de 1990, quando estudos indicavam que os estados unidos estavam caminhando para um colapso empregatício e econômico somados à escassez de profissionais qualificados nas áreas STEM e um alto nível de desinteresse de jovens alunos nessas áreas. Com base nisso, o movimento propõe a reformulação dos métodos de ensino para se adequar à realidade dos alunos, trazendo maior atratividade e incentivando o desenvolvimento de carreiras STEM. Em termos de metodologia, o movimento preza pela aprendizagem baseada em projetos e desafios, estimulando a curiosidade e participação dos alunos. Apesar de ter se difundido com sucesso pelos Estados Unidos e outros países líderes em educação e tecnologia ao redor do globo, o movimento STEM ainda é tímido no Brasil, caminhando a passos curtos com pouco empenho por parte do sistema básico de educação. 

\section{Robótica Educacional}\label{sec:robot_educ}
De  acordo com \cite{nascimento}, a Robótica Educacional, também conhecida como Robótica Pedagógica, é aplicada em ambientes educacionais onde o aluno pode montar e desmontar um robô ou sistema robotizado, proporcionando aos educandos momentos não só de aprendizado, mas de lazer e entretenimento. Esse termo nasceu por volta da década de 1960, através dos estudos de Seymour Papert e sua teoria que defende o uso do computador nas escolas como um recurso atrativo para crianças, e se popularizou entre os jovens a partir da década de 1990 através do movimento STEAM, mas ainda não está bem integrada como uma ferramenta universal de aprendizagem tecnológica em ambientes escolares regulares. 
Segundo \cite{maisonnette}, a robótica educacional é uma ferramenta interdisciplinar de extremo potencial, que extrapola os limites da sala de aula e instiga o aluno a consultar conteúdo e professores de variadas áreas na busca por uma solução para o seu problema; e é através dessa ferramenta que o aluno constrói o conhecimento através das próprias observações e do próprio esforço, adquirindo uma aprendizagem mais efetiva que se adapta a suas estruturas mentais por ser palpável. Parte dessa otimização na maneira de se aprender está no papel do professor, que deve assumir, segundo \cite{nascimento}, o papel de “problematizador” que ajuda o aluno a buscar de maneira autônoma a solução, bem como estreitar o caminho entre o conhecimento empírico e o conhecimento científico. O mesmo autor diz que, para desenvolver o uso da robótica pedagógica, o aluno deverá identificar um problema e entender como solucionar de maneira ordenada utilizando um robô; em seguida, ele desenvolve a programação e testa; ao fim, o aluno pode observar seus resultados e obter a chance de corrigir seus erros caso não atinja os resultados esperados. 
Aderir ao movimento STEM é o primeiro passo para aplicar a robótica educacional nas escolas, no entanto, existe muito conservadorismo e dúvidas quanto a como aplicar uma metodologia efetiva, levando muitos autores a realizarem estudos na área para comprovar a eficiência de determinados métodos. Analisando alguns desses estudos, foi possível verificar um padrão de boas práticas e uma tendência a escolher determinadas metodologias para ensinar robótica.
\subsection{Project-based learning (PBL)}\label{sec:pbl} 
De acordo com \cite{karahoca}, a PBL (em português: aprendizagem baseada em projetos) é uma metodologia de ensino que aumenta a motivação e promove a auto-orientação enquanto o aluno desenvolve e aplica princípios de pensamento crítico, coleta e analisa dados, investiga e aprimora questões, debate ideias, faz previsões e compartilha suas conclusões e descobertas com os demais; podendo ser construída em oito estágios:
\begin{quote}
1) Envolver os alunos em problemas do mundo real; se possível, os alunos selecionam e definem os problemas. O desenvolvimento de robô seguidor de linha é um dos principais desafios.

2) Requerer que os alunos pesquisem, investiguem, usem habilidades de planejamento, pensamento crítico e resolução de problemas enquanto executam tarefas como: colocar materiais no lugar certo, estabelecer posições de momento e equilíbrio, selecionar materiais elétricos, etc.

3) Requerer que os alunos aprendam e apliquem conhecimentos e habilidades de conteúdos específicos em uma variedade de contextos, enquanto trabalham no projeto aprendendo elementos de circuito, soldando, operando com silício de calor, descascando cabos, etc.

4) Oferecer oportunidades para os alunos aprenderem e praticarem habilidades interpessoais enquanto trabalham em equipes, com adultos em locais de trabalho sempre que possível, havendo seleção de liderança, comunicação e distribuição de tarefas.

5) Dar aos alunos a prática de usar o conjunto de habilidades necessárias para suas vidas e carreiras adultas (como alocar tempo / recursos; responsabilidade individual, habilidades interpessoais, aprendizagem através da experiência, etc.). Propõe-se colocar os alunos sob pressão de tempo e materiais limitados, situação essa que  é comum na vida profissional.

6) Incluir desde o início do projeto expectativas em relação a realizações/resultados de aprendizagem por parte dos alunos e padrões e resultados por parte da escola/estado.

7) Incorporar atividades de reflexão que levam os alunos a pensar criticamente sobre suas experiências e vincular essas experiências a padrões específicos de aprendizagem.

8) Terminar com uma apresentação ou produto que demonstre aprendizado e seja avaliado; os critérios podem ser decididos pelos alunos, por exemplo, corrida de seguidores de linha. \end{quote}

Estas etapas estimulam a criatividade, buscas por conhecimento e aprendizagem através de prazer e satisfação enquanto se diverte buscando resultados.

\subsection{Collaborative Learning ou team based learning (TBL)}\label{sec:tbl} 
A aprendizagem colaborativa ou aprendizagem baseada em times é uma situação em que duas ou mais pessoas aprendem ou tentam aprender algo juntos \cite{dillenbourg}. Através dela os alunos aprendem virtudes de colaboração e união, enquanto compartilham e agregam conhecimentos e trabalham juntos na resolução de problemas em equipe, buscando evolução e resultados coletivos e individuais através de liderança, comunicação, cordialidade e distribuição de tarefas. Observa-se com as conclusões do experimento de \cite{karahoca}, que a aprendizagem colaborativa é importante na robótica educacional, pois equipes que exercem atividades com contribuição coletiva se sobressaem a grupos individualistas nos aspectos de desempenho e aprendizagem individual e coletiva. 

\subsection{DIY - Do It Yourself}\label{sec:diy}
O conceito de DIY (Do-It-Yourself), em português: Faça-você-mesmo, é bem comum na atualidade e se popularizou a partir do início do sec. XXI, através das redes sociais, com o intuito de permitir que qualquer pessoa aprenda a construir, consertar, modificar, fabricar e desenvolver os mais diversos tipos de objetos e projetos de maneira objetiva e sem a necessidade de comprar algo pronto ou de contratar um profissional. Este conceito é muito importante de ser aplicado no ensino de robótica porque permite que qualquer pessoa aprenda a desenvolver um robô sem a necessidade de um ambiente escolar, de um professor ou de conhecimentos avançados na área, podendo aprender na própria casa.

\subsection{Movimento Maker}\label{sec:maker} 
O movimento maker é uma extensão da cultura DIY, sendo originado quando a revista Make Magazine, criada nos Estados Unidos, promoveu a Maker Faire (feira de fazedores). Após o sucesso e repercussão do evento, grandes empresas de tecnologia como Samsung, Intel, Microsoft, Raspberry, Arduino e Microchip começaram a desenvolver tecnologias exclusivamente para atender esse público. Na robótica educacional, o Movimento Maker caminha lado a lado com STEM, pois em ambos, a ideia é inovar, empreender e evoluir. De maneira resumida, o movimento Maker valoriza a possibilidade de utilizar das informações obtidas por pesquisa e conteúdos online de fácil acesso para fazer projetos com as próprias mãos, seja com ajuda de um computador, impressora 3D ou ferramentas, visando aumentar a
atratividade para as áreas da ciência e desmistificar tabus relacionados à dificuldade de se aprender novas tecnologias. Como resultado, mais jovens e adultos interessam-se por tecnologia, seguem carreiras na área e aumentam o número de
profissionais qualificados no mercado.

Ainda no campo da pedagogia, o ensino da robótica é interdependente de
aulas no formato da pedagogia clássica, porém melhor aproveitado quando associado a atividades práticas em grupo. Por este motivo considera-se que o movimento Maker e algumas metodologias de ensino de robótica são baseadas na concepção de Lev Vygotsky, a qual diz que o sujeito é considerado um ser não só ativo como também interativo, porque adquire conhecimentos a partir de relações intra e interpessoais, exercitando aquilo que o homem tem de melhor: a criatividade \cite{palangana} e \cite{rocha}. 
Segundo a concepção de Vygotsky, a aquisição de conhecimentos se dá pela interação do sujeito com o meio.



Em todos os experimentos, instituições de ensino e produtos voltados a robótica educacional  analisados neste trabalho, dois ou mais dos conceitos acima descritos foram aplicados, demonstrando um padrão de métodos efetivos para ensino de robótica.


\begin{flushright}
	"Elementar , meu caro Watson." \\
	\ \\
	(Scherlock Holmes)
\end{flushright}

O termo robô vem da palavra tcheca robota que tem como uma das possíveis traduções “trabalhador forçado” e ganhou o significado atual após o escritor tcheco Karel Capek (1809 - 1938), na sua obra de ficção científica “R.U.R. Rossumovi Univerzální Roboti”, associar o termo às máquinas criadas pelo personagem principal para servi-lo. Mas a ideia de algo que desenvolva atividades de maneira autônoma é apresentada ao mundo muito tempo antes. \cite{aristoteles1985traduccao} diz: “Se cada instrumento pudesse realizar sozinho a sua tarefa, obedecendo ou antecipando a nossa vontade, [...] os feitores não precisariam de servos, nem os senhores de escravos.” 

Diversas obras da ficção retratam diferentes tipos de robôs criados de forma a reproduzir comportamentos semelhantes aos de um ser humano. Com o passar do tempo, juntamente com o avanço tecnológico nas áreas da eletrônica, mecânica e informática, a construção dessas máquinas se tornou possível. A indústria observou nos robôs, o potencial para automatizar e otimizar as linhas de processo, onde atividades que pudessem demandar mais tempo se fossem executadas por seres humanos, seriam executadas de forma muito mais rápida e precisa com a utilização de máquinas programadas e autônomas, aumentando a produção.

A \cite{iso2012} define um robô como “mecanismo programável atuado em dois ou mais eixos com um grau de autonomia, movendo-se dentro do seu ambiente, para executar tarefas pretendidas”. É resultado da integração de componentes como: Sensores; atuadores; unidade de controle; unidade de potência e manipulador mecânico. Sensores são os componentes que fornecem parâmetros sobre o ambiente em que o robô se encontra e sobre o comportamento do próprio sistema robótico. Já os atuadores são os dispositivos que movimentam as partes, quando convertem energia elétrica, hidráulica ou pneumática em mecânica. A energia necessária para o funcionamento dos atuadores é fornecida pela unidade de potência.

O gerenciamento dos parâmetros necessários para que o robô realize suas tarefas é de responsabilidade da unidade de controle. De onde também são emitidos os comandos para a movimentação. O manipulador mecânico é o conjunto de componentes estruturais do robô, elos ou links, conectados entre si por articulações comumente denominadas de juntas. Graus de liberdade, segundo \cite{romanorobotica} “É o número mínimo de variáveis independentes de posição que precisam ser especificadas para se definir inequivocamente a localização de todas as partes de um mecanismo”.

\section{Cinemática}\label{sec:cinem}
A cinemática é o ramo da física que descreve o movimento de um corpo, determinando características como posição, velocidade e aceleração. Na robótica, o estudo cinemático resulta em um conjunto de equações que caracterizam o movimento do robô, a complexidade da solução varia com a quantidade de graus de liberdade que esse robô tem. Em um manipulador mecânico composto por links que são conectados por juntas, cada conjunto link-junta caracteriza um grau de liberdade. Dessa maneira, um robô com $n$ conjuntos link-junta tem $n$ graus de liberdade, sendo o primeiro link a base de sustentação do robô no mundo e o último, onde está a seu end-effector.

\subsection{Cinemática Direta}\label{sec:cinem_dir}
A cinemática direta é a solução para a movimentação de um robô com cálculo da posição e orientação do end-effector a partir de dadas posições das juntas. A notação de Denavit-Hartenberg é uma ferramenta utilizada para coordenar a descrição cinemática de sistemas mecânicos articulados com $n$ graus de liberdade.

\begin{figure}[h!]												
	\centering												
	\includegraphics[width=0.5\textwidth]{denavit.PNG}			
	\caption{Parâmetros de Denavit-Hartenberg}		
	\label{img:denavit}	
	\source{\cite{romanorobotica}}		
\end{figure}

A figura mostra dois \textit{links} ligados por uma junta de superfícies deslizantes uma sobre a outra. Um eixo de uma junta estabelece a conexão de dois \textit{links}. Segundo \cite{romanorobotica}, os eixos das juntas devem ter duas normais conectadas a eles, uma para cada um dos \textit{links}. Assim a posição relativa destes dois \textit{links} conectados ($i-1$ e $i$) é dada por $d_{i}$, que é a distância medida ao longo do eixo da junta entre suas normais. O ângulo de junta $\theta_{i}$ entre as normais é medido em um plano normal ao eixo da junta. Dessa forma, $d_{i}$ e $\theta_{i}$ são a distância e o ângulo entre os \textit{links} adjacentes. Determinam a posição relativa de \textit{links} vizinhos.

Um \textit{link} pode apenas ser conectado a dois outros \textit{links} ($i-1$ e $i+1$). Assim, dois eixos de juntas são estabelecidos em ambos terminais de conexão. Os \textit{links} mantém uma configuração fixa entre as juntas e podem ser caracterizados pelos parâmetros $a_{i}$ e $\alpha_{i}$. O parâmetro $a_{i}$ é a menor distância medida ao longo da normal comum entre os eixos da junta, chamado de comprimento de \textit{twist}, já o $\alpha_{i}$ é o ângulo de \textit{twist}. Esses quatro parâmetros determinam a estrutura do \textit{link}, parâmetros da junta e a posição relativa aos \textit{links} vizinhos.

A representação de Denavit-Hartenberg \cite{denavit1955kinematic} tem como resultado uma matriz 4 x 4 representando cada sistema de coordenadas do \textit{link} na junta em relação ao \textit{link} anterior. Essa matriz é obtida através do produto das transformações: Translação de uma distância $d_{i}$ ao longo do eixo $Z_{i-1}$ para trazer os eixos $X_{i-1}$ e $X_{i}$ na coincidência; Rotação no eixo $Z_{i-1}$ de um ângulo $\theta_{i}$ para alinhar os eixos $X_{i-1}$ e $X_{i}$; Translação ao longo do eixo $X_{i}$ de uma distância $a_{i}$ para trazer as duas origens na coincidência; Rotação do eixo $X_{i}$ um ângulo $\alpha_{i}$ para trazer os dois sistemas de coordenadas na coincidência. Isso resulta na matriz de transformação homogênea $^{i-1}A_{i}$.

\begin{equation}
^{i-1}A_{i}=T_{z,d} T_{z,\theta} T_{x,a} T_{z,\alpha}
\end{equation}
\begin{equation}
^{i-1}A_{i}=\begin{bmatrix}
1 & 0 & 0 & 0\\ 
0 & 1 & 0 & 0\\ 
0 & 0 & 1 & d_{i}\\ 
0 & 0 & 0 & 1
\end{bmatrix}\begin{bmatrix}
cos\theta_{i} & -sin\theta_{i} & 0 & 0\\ 
sin\theta_{i} & cos\theta_{i} & 0 & 0\\ 
0 & 0 & 1 & 0\\ 
0 & 0 & 0 & 1
\end{bmatrix}\begin{bmatrix}
1 & 0 & 0 & a_{i}\\ 
0 & 1 & 0 & 0\\ 
0 & 0 & 1 & 0\\ 
0 & 0 & 0 & 1
\end{bmatrix}\begin{bmatrix}
1 & 0 & 0 & 0\\ 
0 & cos\alpha_{i} & -sin\alpha_{i} & 0\\ 
0 & sin\alpha_{i} & cos\alpha_{i} & 0\\ 
0 & 0 & 0 & 1
\end{bmatrix}\end{equation}
\begin{equation}
^{i-1}A_{i}=\begin{bmatrix}
cos\theta_{i} & -cos\alpha_{i}sin\alpha_{i} & sen\alpha_{i}sin\theta_{i} & a_{i}cos\theta_{i}\\ 
sin\theta_{i} & cos\alpha_{i}cos\theta_{i} & -sin\alpha_{i}cos\theta_{i} & a_{i}sin\theta_{i}\\ 
0 & sin\alpha_{i} & cos\alpha_{i} & d_{i}\\ 
0 & 0 & 0 & 1
\end{bmatrix}
\end{equation}

\subsection{Cinemática Inversa}\label{sec:cinem_inv}
Segundo \cite{fu1987robotics} os robôs estão em um espaço onde o objeto a ser manipulado tem sua posição expressa no sistema de coordenadas do ambiente. Com o objetivo de controlar a posição e orientação do \textit{end-effector} do robô, a solução da cinemática inversa é mais adequada. A cinemática inversa consiste em, partindo de uma posição e orientação desejada, calcula-se as posições das juntas para que o robô alcance esse objetivo, é o processo inverso da cinemática direta. 

Há de se observar que a cinemática inversa pode ou não ter solução, caso a posição de interesse esteja fora do espaço de trabalho do robô, não há posições de juntas que execute a tarefa. Nos momentos em que a posição desejada pode ser alcançada, podem existir mais de uma solução. Um ponto importante na solução da cinemática inversa é, quando há mais de uma solução deve-se atentar para qual delas é a melhor opção, levando em consideração o ambiente em que o robô se encontra, principalmente os obstáculos à sua volta. A demanda energética para a execução dos possíveis movimentos e o esforço qual as juntas serão submetidas nesta ação, é crucial para o planejamento da movimentação do robô.


\section{Modelagem Cinemática de um Braço Planar}\label{sec:brac_plan}
O robô ELIR tem na sua estrutura, braços que se movimentam apenas em dois eixos, $x$ e $z$, através da atuação de duas juntas, podendo assim ser modelado cinematicamente como um braço planar do tipo RR. A figura a seguir mostra um exemplo desse braço, RR por ter duas juntas rotativas, que se movimenta no plano $x-y$:

\begin{figure}[h!]												
	\centering												
	\includegraphics[width=0.5\textwidth]{planar_rr.PNG}			
	\caption{Braço planar do tipo RR}		
	\label{img:planar}	
	\source{\cite{cinem_inv}}		
\end{figure}

Usando a análise da cinemática direta, consegue-se determinar a posição do \textit{end-effector} com base nos ângulos $\theta_{1}$ e $\theta_{2}$ e nas dimensões $L_{1}$ e $L_{2}$. Logo tem-se que:
\begin{equation}
x=L_1cos{(\theta_1)}+L_2cos{(\theta_1 + \theta_2)}
\end{equation}
\begin{equation}
y=L_{1}sen{(\theta_1)}+L_{2}sen{(\theta_1 + \theta_2)}
\end{equation}
Aplicando a lei dos cossenos ao triângulo formado pelo braço e pela linha entre a origem do braço e o seu \textit{end-effector} obtém-se: 
\begin{equation}
\theta_2=\pm arccos{\frac{(x^2+y^2-(L_1)^2-(L_2)^2)}{2L_1L_2}}
\end{equation}
Para determinar o $\theta_{1}$ considera-se a relação trigonométrica:
\begin{equation}
tan{(A - B)}=\frac{tan(A)-tan(B)}{1+tan(A) tan(B)}
\end{equation}
e tomando:
\begin{equation}
tan(\beta)=\frac{L_2 sin\theta_2}{L_1 + L_2 cos\theta_2}
\end{equation}
tem-se que:
\begin{equation}
\theta_1 = arctan[\frac{y(L_1 + L_2 cos\theta_2)-xL_2 sin\theta_2}{x(L_1 + L_2 cos\theta_2)-yL_2 sin\theta_2}]
\end{equation}
Assim é possível fazer a solução da cinemática inversa para um braço planar RR.

\section{Desenvolvimento de Robôs}\label{sec:desen_robo}
Para o desenvolvimento de sistemas robóticos, é necessária a integração de vários dispositivos, assim sendo necessário utilizar ferramentas e tecnologias que poupem tempo no desenvolvimento, de forma a facilitar o processo de comunicação entre as diversas camadas de abstração. 
As camadas de abstração se referenciam ao alto e baixo nível da máquina, onde baixo nível é uma referência para aplicações mais simples, que estão mais próximas da linguagem da máquina, como por exemplo aplicação de comunicação somente via \textit{bytes}. Um exemplo de um elemento que está numa camada de abstração de alto nível é uma Interface Homem-Máquina , onde o usuário consegue interagir com a máquina diretamente, sem ter que necessariamente entender o seu funcionamento interno.

\subsection{\textit{Framework}}\label{sec:framework}
Em ambientes computacionais, a utilização de ferramentas para realização de atividades e desenvolvimento de soluções é de extrema importância. Estas ferramentas podem ser softwares específicos para execução de uma determinada atividade ou \textit{frameworks}.	
Segundo \cite{maxwel_framework} “\textit{Frameworks} são estruturas de classes que constituem implementações incompletas que, estendidas, permitem produzir diferentes artefatos de software”. Os \textit{frameworks} em geral permitem o desenvolvimento de soluções computacionais baseadas em determinadas funcionalidades, seguindo uma estrutura definida pelo \textit{framework}. De acordo com \cite{maxwel_framework} os \textit{frameworks} definem uma arquitetura para um conjunto de subsistemas, dando os construtores necessários para a sua criação.

A principal característica de um \textit{framework} é a sua capacidade de reutilização, afinal a sua utilização permite que diversos conjuntos de produtos possam ser gerados partindo de uma única estrutura que possua os conceitos mais gerais.
Segundo \cite{maxwel_framework} \textit{frameworks} podem ser classificados em dois tipos principais: \textit{Frameworks} de Aplicações Orientado a Objetos e \textit{Frameworks} de Componente.
Os \textit{frameworks} orientados a objetos geram famílias de aplicações orientadas a objetos e seus pontos de extensão são definidos como classes abstratas ou interfaces, onde se estendem por cada instância da família de aplicações.
Para \textit{frameworks} de componentes, o suporte é previsto para componentes que sigam um determinado modelo, possibilitando que as instâncias destes componentes sejam acopladas ao \textit{framework}. Também são estabelecidas as condições necessárias para que um componente seja executado, regulando a sua interação entre as instâncias de outros componentes.

Os \textit{frameworks} utilizados para robótica, são extremamente importantes, pois o uso de suas ferramentas possibilita o desenvolvimento e criação das soluções computacionais e códigos necessários para cada funcionalidade de um robô, de forma que o funcionamento delas em conjunto seja otimizado pela natureza do \textit{framework} de realizar a compatibilização entre as estruturas.

\subsection{Simulação}\label{sec:simula}

Em sistemas complexos, onde diversas variáveis definem o seu funcionamento, e por consequência as suas respostas a determinados estímulos, torna-se extremamente difícil e irresponsável executar a sua fabricação antes de realizar uma validação prévia de seu funcionamento.

Este tipo de procedimento de análise prévia do comportamento de um sistema é chamado de simulação. De acordo com \cite{definicao_de_simulacao} "Simulação refere-se a uma ampla coleção de métodos e aplicações para imitar o comportamento do sistema real, por meio de um computador com um \textit{software} apropriado". Diversos tipos de sistemas se utilizam da ferramenta de simulação para validar previamente o funcionamento de projetos.

O processo de utilização de uma simulação consiste em basicamente recriar o sistema em questão em um ambiente computacional e então são fornecidas as entradas para o sistema, as rotinas de tratamento destas entradas e por fim as saídas da simulação.

\begin{figure}[h!]												
	\centering												
	\includegraphics[width=0.5\textwidth]{simulation.PNG}			
	\caption{Diagrama de funcionamento de um processo de simulação}		
	\label{img:simulation}	
	\source{\cite{definicao_de_simulacao}}		
\end{figure}


Em sistemas robóticos, uma ferramenta extremamente útil e bastante utilizada, é a simulação. Segundo \cite{artigo_sobre_simulacao} 
\begin{quote}
“Quando se trabalha com robótica, o uso de uma simulação é de importância significante. Por um lado, ela permite a validação de diferentes alternativas durante o design do sistema robótico, levando assim, a melhores decisões e preservação de custos. Por outro lado, auxilia o processo de desenvolvimento de \textit{software}, disponibilizando uma reposição para robôs que não estejam em mãos”.
\end{quote}

Através do uso de \textit{softwares} de simulação é possível criar uma representação computacional não só o modelo físico de um robô, mas também os parâmetros referentes à objetos do ambiente no qual o mesmo será posto em funcionamento. A avaliação prévia da execução das tarefas e do funcionamento do robô, permite a observação do comportamento do sistema em determinadas situações, facilitando assim, a tomada de decisões mais efetivas no processo de desenvolvimento do protótipo real. 

\subsection{Odometria}\label{sec:odom}
A odometria consiste no cálculo para estimar a mudança de posição do robô no tempo, onde isso pode se dar por meio de diversos dispositivos que possibilitem o cálculo de deslocamento. Onde segundo \cite{ben2018robotic}, "Odometria - a medição da distância - é um método fundamental usado por robôs para navegação". A medição de tempo é fácil utilizando o \textit{clock} interno do computador embutido. Medir velocidade é mais difícil: em alguns robôs educacionais utilizam codificadores são usados para contar as rotações da rodas, enquanto em outros a velocidade é estimada das propriedades dos motores.

No caso da análise de deslocamento do robô na linha por meio de roldanas, o movimento é caracterizado como linear, já que o deslocamento ocorre em somente uma direção, analogamente a odometria utilizada é a linear, onde o deslocamento pode ser calculado simplesmente pela equação \ref{eq:deslocamento} onde $s$ representa o espaço caminhando, $v$ a velocidade e $t$ o tempo. 

\begin{equation}\label{eq:deslocamento}
s = v*t
\end{equation}

Utilizando o medidor de tempo interno do computador embutido nos sistemas robóticos, pode se calcular a variação de espaço para um tempo muito pequeno, onde esses pequenos incrementos são somados ou subtraídos para encontrar o deslocamento do robô.

A velocidade de deslocamento das roldanas pode ser encontrada utilizando a equação \ref{eq:vel} com as informações do raio da roldana $r$ e a sua velocidade de giro $w$ em radianos por segundo. A informação da velocidade de giro da roldana geralmente é extraída dos servomotores utilizados para tração.

\begin{equation}\label{eq:vel}
v = 2\pi*r*w
\end{equation}

O cálculo da odometria por meio da velocidade das rodas é denominado no âmbito da robótica como odometria de roda, \textit{wheel odometry} em inglês, porém, outras técnicas são utilizadas, já que existem diversos tipos de deslocamento diferentes. Outro tipo aplicação muito encontrada é a odometria visual, que segundo \cite{nister2004visual} "Odometria Visual (OV) é o processo de estimação do deslocamento de um agente (ex: veículo, humano e robô) utilizando a entrada de uma ou múltiplas câmeras conectadas a ele". Os domínios da aplicação incluem robótica, realidade aumentada, automotiva e 'computadores vestíveis'.

\subsection{Gestão de Energia}\label{sec:gestao}
O conceito de gestão de energia se dá pela forma como a energia elétrica é utilizada em um sistema composto de diversos dispositivos elétricos e eletrônicos. Para sistemas robóticos, este conceito representa um fator importante para garantir uma operação autônoma de qualidade. Os robôs quando nesse tipo de operação, geralmente não dispõem de uma fonte de energia constante, e portanto, são geralmente alimentados por baterias e tendo interação por meio de conexões sem fio. 

O uso de diversos dispositivos eletrônicos de baixo consumo energético, como sensores e interfaces microcontroladas, podem não se mostrar um problema para um curto período de operação, porém, para maximizar o tempo da atividade exercida pelo robô, é necessário encontrar uma forma eficiente de gerir a operação dos dispositivos conectados na rede de alimentação. Segundo \cite{katiraei2006power} 
\begin{quote}
“Gestão de energia é um conceito importante em redes de sensores, porque uma estrutura de energia cabeada geralmente não está disponível e um conceito óbvio é utilizar a energia disponível da bateria de forma eficiente”. 
\end{quote}	
Quanto mais atividades diferentes o robô desempenha maior será a demanda de energia entre os dispositivos interconectados, isso faz com que seja necessário que os desenvolvedores busquem uma forma de otimizar o custo de energia individual das atividades e do fluxo de operação como um todo. Os diversos dispositivos utilizados em sistemas robóticos fazem com que o mesmo se utilize de diferentes níveis de tensão e corrente, já que comumente, os dispositivos utilizados são comerciais, e devido às diferenças das suas características e parâmetros, definidos por empresas diferentes, responsáveis pela produção e fabricação das ferramentas, é necessário que a gestão de energia leve em consideração a compatibilidade entre diferentes dispositivos.

\subsection{Conceito de segurança e Integridade}\label{sec:segur_inte}
Em diversas áreas, é comum a verificação das condições antes da execução de atividades, a aviação é um grande exemplo de uso desse conceito. Neste seguimento, o \textit{checklist} é utilizado toda vez antes de um avião decolar, assim é possível verificar se os sistemas vitais para o vôo estão em ordem. O principal objetivo dessa ação é identificar os riscos que existem para o cumprimento da atividade.

Para que um dispositivo robótico execute as tarefas para as quais ele foi desenvolvido, deve-se verificar se os seus sistemas, como um todo, e os componentes individualmente, estão em condições de funcionamento, garantindo assim a integridade do sistema como um todo. Essa análise deve ser feita levando em conta a importância de cada sistema e de cada componente desses sistemas, a fim de aumentar a capacidade de operação em condições adversas do robô. Podem existir sistemas que, mesmo quando não estão operando adequadamente, não comprometem a execução da missão do robô.

\subsection{Comunicação em sistemas robóticos}\label{sec:comm_sis}
Dispositivos eletrônicos são capazes de realizar transmissão de dados, afinal, a interconexão entre eles é de extrema importância em sistemas em que existam diversos dispositivos responsáveis por funções distintas. Dispositivos que se comunicam entre si, são capazes de criar uma rede em todo o sistema, permitindo um aumento na confiabilidade das funções do sistema, através da troca de informações de parâmetros que venham a ser importantes para o funcionamento do sistema como um todo.

Para que os dispositivos possam se comunicar entre si, os mesmos adotam o que se chama de protocolos de comunicação. Protocolos de comunicação são arquiteturas que estabelecem a troca de dados entre dispositivos eletrônicos. Os dispositivos comerciais possuem diferentes tipos de protocolos de comunicação e por isso, torna-se extremamente importante se atentar a qual protocolo utilizar durante a conceituação de um projeto que se tenha a necessidade da interconexão de dispositivos.

Uma das formas mais comuns de se realizar a transmissão de dados entre dispositivos embarcados é a comunicação serial. \cite{livro_sistemas_embarcados} define a comunicação serial como um envio de \textit{bits} de forma serial, similar a uma fila. Possuindo dois canais principais: o canal $TX$ para envio e o canal $RX$ para recebimento. Dentro desse processo de comunicação alguns parâmetros devem ser levados em conta, como a taxa de transmissão de dados (\textit{BaudRate}); bits de paridade, para assegurar que o número de bits no campo de dados é par ou ímpar; bits de parada para indicar o início ou fim de uma comunicação.

Outro meio de comunicação muito utilizado é o USB (\textit{Universal Serial Bus}), criado com a intenção de tornar a comunicação serial mais simplificada e com uma taxa de transmissão muito mais elevada. Os cabos conectores USB possuem geralmente quatro fios condutores, sendo dois deles para alimentação e dois outros cabos de dados. Os cabos de dados são nomeados como D+ e D-, onde a comunicação entre os dispositivos se dá pela variação de tensão entre estes dois sinais.
Dentro de ampla complexidade como um robô, onde diversos dispositivos necessitam estar trocando informações, a utilização de protocolos de comunicação serial se tornam extremamente importantes para a garantia da confiabilidade na execução de tarefas e operações.
