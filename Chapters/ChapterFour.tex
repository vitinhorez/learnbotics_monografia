\chapter{Desenvolvimento}
\label{chap:desen_test}
\begin{flushright}
	"Sou eu, ou o mundo está ficando cada vez mais louco?" \\
	\ \\
	(Coringa)
\end{flushright}

Para obter de forma efetiva o proposto na metodologia, foram elaborados dois conjuntos principais de entregáveis: os Tutoriais, e o Kit Físico. Após finalização do projeto, os Tutoriais se encontram em domínio virtual, na Wiki do repositório da Learnbotics no Github, e o Kit Físico foi fabricado e montado. A seguir encontra-se uma melhor explicação de como esses resultados foram obtidos.

\section{Preparação do Hardware}
Para que tudo o que foi pensado pelo projeto pudesse ser desenvolvido e passado ao usuário, uma escolha cuidadosa do hardware foi efetuada. Posteriormente, uma configuração e testes deste hardware foram também realizados.

\subsection{Componentes de Hardware}

\subsection{Configuração da Raspberry Pi}

\subsection{Componentes de Software}

\subsection{Testes de Hardware e Software}

\section{Tutoriais}
De forma a melhor organizar a elaboração do conteúdo dos tutoriais, uma divisão dos mesmos foi feita de acordo com o assunto abordado, ocasionando assim uma menor abrangência de assuntos a serem pesquisados, de conteúdos a serem concentrados e de novas interpretações a serem elaboradas. As subseções a seguir tratarão de partes específicas dos tutoriais.

\subsection{Um Breve Histórico da Robótica}
siaufgasiogfçwaugfsçfçsuaigfawgfçksgfiauçgwfgsfsaf fsiuafawfioashf saif sfiaywo fasfhaufaw fsifshaf wif saif w

\subsection{Introdução à Robótica atual e Alguns Conceitos Básicos}

\subsection{Introdução à Atuação}
Para que o estudante possa compreender de uma forma mais aprofundada o que está fazendo, antes de começar a mexer com os Dynamixels, uma pequena introdução a atuação foi elaborada.

Essa Introdução trata de uma forma simplificada do conceito de atuadores, de tipos de atuadores, de conversão de energia, e apresenta exemplos cotidianos de atuadores explicando seu funcionamento e aprofundando um pouco mais os conceitos de conversão de energia.

COLOCAR IMAGEM DA WIKI???? 

\subsection{Introdução à Visão Computacional}

\subsection{Tutoriais da Raspberry Pi}

\subsection{Tutoriais dos Dynamixels}
Após ter tido contato com o conceito de atuadores, o estudante irá encontrar também um tutorial que faz uma introdução aos servo-motores inteligentes Dynamixel$^{TM}$.

Neste tutorial são apresentados os servo-motores inteligentes, suas diferenças para servo-motores comuns, suas vantagens sobre os comuns, qual o papel destes servo-motores no robô e no kit físico e mais precisamente porquê escolhemos utilizar os Dynamixels, e não servo-motores comuns.

COLOCAR IMAGEM DA WIKI????


\subsection{Tutoriais do ROS}
Tendo em vista a ídeia de apresentar ao estudante ferramentas que são de fato utilizadas por profissionais da área, buscamos realizar um material completo sobre as partes iniciais de utilização do framework ROS.

Devido ao nível de conteúdo que é abordado nos tutoriais nativos do ROS, foi feita uma análise de relevância dos conteúdos e uma reescrita completa do conteúdo abordado, trazendo novas abordagens para passar esse conhecimento para o estudante.

Este tutorial foi dividido em quatro partes, sendo elas, em ordem:
\begin{itemize}
	\item Introdução: O que é o ROS e como funciona;
	\item Conceitos Básicos: Apresentação de terminologia e conceitos base utilizados pela comunidade do ROS.
	\item Entendendo como Funciona o ROS: Apresentação de conteúdo novo que foi elaborado com base em analogias para facilitar o entendimento do estudante sobre a ferramenta.
	\item Tutoriais do ROS: Os Tutoriais de fato, onde o aluno irá aprender a configurar e utilizar o ROS.
\end{itemize}

A parte quatro, ou parte dos tutoriais de fato, aborda todos os conceitos de nível iniciante apresentados nos tutoriais oficiais do ROS. Porém estes conceitos foram demonstrados de forma mais simplificada, com linguagem mais simples e de forma acompanhada passo a passo para uma melhor assimilação do estudante.

COLOCAR IMAGEM DA WIKI????

\subsection{Apresentação dos Scripts de Cinemática}
Nesta parte dos tutoriais o estudante terá acesso ao programa que fará com que o seu robô ande. Além disso será ensinado também como o estudante deve proceder para que transforme seu código em um código executável e para rodá-lo.

De forma a estimular o estudante, uma análise mais minusciosa do código, com comentários parte a parte também foi feita. A partir da explicação do que os comandos do programa fazem o estudante terá condições de alterá-lo para concluir etapas e descobrir coisas por si só.

Para finalizar, o tutorial apresenta um desafio para que o estudante de fato assimile o que lhe está sendo proposto, alterando o código e vendo na prática o que isso ocasiona.

COLOCAR IMAGEM DA WIKI????


\subsection{Introdução ao OpenCV}

\subsection{Apresentação dos Scripts de Visão Computacional}

\subsection{Integração dos assuntos anteriores}

\subsection{Desafio Final}

\subsection{Tutoriais de Montagem do Robô}

\section{Kit Físico}

\subsection{Design}

\subsection{Fabricação}

\subsection{Montagem}
 



