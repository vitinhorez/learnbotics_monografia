% !TeX spellcheck = pt_BR
\chapter{Desenvolvimento}
\label{chap:desen_test}
\begin{flushright}
	"Insanidade é continuar fazendo sempre as mesmas coisas, \\ 
	esperando resultados diferentes." \\
	\ \\
	(Albert Einstein)
\end{flushright}

Durante cada uma das etapas da metodologia, uma série de tarefas foram elaboradas e cumpridas. Este capítulo irá tratar do desenvolvimento de cada dessas etapas de acordo com a metodologia utilizada na realização deste projeto.

\section{Etapa Conceitual}
\subsection{Levantamento de Requisitos}
\subsection{Estudo do Estado da Arte}
\subsection{Estudo das Metodologias de Ensino}
\subsection{Finalização do Conceito}

\section{Projeto Detalhado}
\subsection{Definição}
Após a realização das etapas descritas anteriormente, pôde-se de fato escrever uma solução proposta. A solução proposta foi dividida em duas partes, o Kit Físico e os Tutoriais, como descrito a seguir.

Kit Físico: Utilização da Raspberry Pi como controlador central, utilização dos dynamixels mx-28 como atuadores, kit modulado com módulos integrados e sequenciais para montagem de um robô reconhecedor de marcos fiduciais. A finalização do Kit culminará em um robô com movimentação diferencial e capacidade de visão com a utilização de uma câmera RGB.

Tutoriais: Linguagem simples e acessível, metodologia intuitiva, presença de desafios.
Conteúdos a serem abordados: Breve histórico da robótica, partes de um robô, funcionalidades encontradas em robôs completos, introdução a áreas da robótica, abordagem prática do framework ROS em compatibilidade com a Raspberry Pi, abordagem prática de conceitos de programação em Python com a utilização de programas modelos acompanhados de tutoriais de mudança e proposição de desafios,
programação de atuação dos servos para movimentação diferencial simples, abordando conceitos facilitados de cinemática diferencial e seu funcionamento. Introdução à visão computacional com introdução a biblioteca do OpenCV e tratamentos simples de imagens.
Implementação de uma integração da movimentação com a visão computacional.
\subsection{Planejamento}

\section{Confecção}
\subsection{Protótipo Físico}
\subsection{Material Escrito}
\subsection{Desafios}

