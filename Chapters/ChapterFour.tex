% !TeX spellcheck = pt_BR
\chapter{Desenvolvimento}
\label{chap:desen_test}
\begin{flushright}
	"Insanidade é continuar fazendo sempre as mesmas coisas, \\ 
	esperando resultados diferentes." \\
	\ \\
	(Albert Einstein)
\end{flushright}

Durante cada uma das etapas da metodologia, uma série de tarefas foram elaboradas e cumpridas. Este capítulo irá tratar do desenvolvimento de cada dessas etapas de acordo com a metodologia utilizada na realização deste projeto.

\section{Etapa Conceitual}
\subsection{Levantamento de Requisitos}
O primeiro passo para o desenvolvimento do projeto foi o de reunir todos os requisitos que deveriam ser cumpridos pelas entregas. A partir de conversas com o cliente, pode-se levantar alguns requisitos iniciais.

As funcionalidades esperadas pelo cliente foram um kit físico dividido em módulos integrados e complementares. O kit deveria culminar na montagem de um robô com movimentação cinemática e funcionalidade de visão computacional utilizando uma câmera RGB.
 
Além disso o cliente esperava também tutoriais online abrigados em domínio aberto, escritos em linguagem simplificada, abordando conceitos introdutórios da robótica, como por exemplo: O que é um robô, partes de um robô, funcionalidades de um robô completo, áreas da robótica e suas funcionalidades, atuação e movimentação diferencial, introdução a visão computacional e integração com a movimentação.

Um outro ponto exigido pelo cliente foi a utilização de desafios ao longo do desenvolvimento através do kit, com o intuito de manter o estudante engajado e interessado durante a sua interação com o kit.

Por fim o cliente pediu ainda que fossem abordados conceitos de ferramentas utilizadas profissionalmente, em específico o framework ROS, alguma linguagem de programação e alguma biblioteca para auxilio de aprendizado na questão da visão computacional.

\subsection{Estudo do Estado da Arte}
\subsection{Estudo das Metodologias de Ensino}

\section{Projeto Detalhado}
\subsection{Definição}
\subsection{Planejamento}

\section{Confecção}
\subsection{Protótipo Físico}
\subsection{Material Escrito}
\subsection{Desafios}

