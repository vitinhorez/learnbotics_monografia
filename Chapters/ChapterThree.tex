\chapter{Metodologia}
\label{chap:meto}
\begin{flushright}
	"Tudo o que temos de decidir é o que fazer com o tempo que nos é dado." \\
	\ \\
	(Gandalf)
\end{flushright}

\section{Metodologia de Ensino}\label{sec:metod_ensino}
Visando facilitar a implementação e integração do ensino de robótica no Brasil, foi desenvolvido um kit de robótica pedagógica para ser aplicado em ambientes educacionais e extrapolar os limites da sala de aula, permitindo que os conceitos de robótica possam ser aprendidos pelos entusiastas em qualquer ambiente. A metodologia de ensino foi baseada nos requisitos fundamentais de robótica educacional, a fim de proporcionar aos alunos momentos de aprendizado, lazer e entretenimento. Para tal, foi realizado um estudo das principais e mais eficazes didáticas de ensino de robótica ao redor do mundo, a fim de reunir os métodos mais utilizados e comprovadamente efetivos para aplicar no projeto. Desta maneira, foram selecionadas as seguintes metodologias de aprendizado: o aluno deve ser motivado a desenvolver pensamento crítico, analisar soluções e superar desafios através da metodologia PBL; o kit de aprendizado pode ser utilizado e montado em equipe para propiciar o aprendizado coletivo e o desenvolvimento de virtudes como união e trabalho em equipe, características fundamentais para os profissionais do futuro, através da metodologia TBL; a metodologia DIY está presente permitindo que qualquer pessoa possa se interessar e aprender com as próprias ações sem precisar de um ambiente escolar ou de um professor; e a participação no movimento maker ao desmistificar o mundo da robótica e atrair novos entusiastas através de linguagem simples e atrativa. De todos os trabalhos mais relevantes na área de ensino de robótica, ao menos duas das características selecionadas para o projeto estavam presentes.
	