\chapter{Conclusão}
\label{chap:conc}
\begin{flushright}
	"Só sei que nada sei." \\
	\ \\
	(Sócrates)
\end{flushright}

Este trabalho apresentou as prerrogativas e decisões envolvidas na proposição de uma nova abordagem para o ensino da robótica nos níveis não graduados, tal como as metodologias utilizadas, as soluções desenvolvidas e os resultados alcançados durante a execução deste projeto.

Pode-se concluir que o projeto foi finalizado apresentando todos os entregáveis discutidos com o cliente, uma vez que todos os assuntos foram abordados e dispostos no github, além de todas as ferramentas requisitadas terem sido incluídas nas soluções apresentadas.

O conteúdo teórico foi escrito em linguagem acessível, e disposto em formato de tutoriais e apostilas disponíveis em domínio virtual. O foco destes tutoriais é apresentar conceitos de uma forma simples, direta e utilizando uma linguagem informal, porém apresentando termos técnicos utilizados no mundo da robótica. Ademais, todos os tutoriais estarão disponíveis no Github, em formato de wiki, em repositório aberto, escritos em português, de forma a promover a acessibilidade do conteúdo à comunidades lusófonas em geral. Vale salientar que atualmente há 24 páginas no github do Learnbotics, apresentando conteúdos diversos como: Visão computacional, cinemática, o que é a robótica etc. Nas páginas, pode-se encontrar 10 programas modelos que ajudarão aos futuros alunos a assimilarem melhor o conteúdo.

O kit físico foi dividido em módulos complementares de montagem, resultando em um robô com movimentação diferencial. Por se tratar de um robô simples, que foi pensado para ser fabricado através de manufatura aditiva, apresenta fácil montagem englobando todos os componentes do kit físico e tangenciando conceitos de poka yoke, a fim de propiciar ao usuário um aprendizado mais amigável. Este kit se torna um diferencial quando estimula o aluno a buscar uma maior interação com a robótica ao passo em que exercita a sua criatividade através de objetos físicos que interagem com conceitos abstratos.

Esta modularização será atrelada à progressão do aluno. Na primeira parte ele terá acesso, principalmente ao computador, aprendendo seu funcionamento básico. Na segunda parte terá acesso aos servomotores e aprenderá a conectá-los ao computador e a enviar comandos a partir de scripts. O próximo passo será adicionar a webcam e aprender a utilizar ferramentas de visão computacional. O estudante continuará tendo acesso a módulos complementares passo a passo a medida que vai avançando nos conceitos, até que chegue ao desafio final que irá integrar todos os passos apresentados anteriormente.

Ao combinar metodologias de ensino diferentes, focando principalmente na parte prática do aprendizado e visando dar forma e visualização a conceitos muitas vezes estritamente abstratos, essa nova abordagem de aprendizado apresenta características apelativas à um público mais jovem. Algumas dessas características que podem ser identificadas são a utilização de uma linguagem mais acessível, e um estímulo à criatividade e ao desenvolvimento de habilidades práticas.

Ao aplicar ferramentas que são utilizadas por profissionais da área, como por exemplo um sistema operacional baseado em Linux, o framework ROS e plataforma de versionamento como o Github, o estudante terá desde cedo contato com conceitos e habilidades requisitadas pelas funções desempenhadas por um profissional da área de robótica.